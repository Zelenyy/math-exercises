\documentclass[A4paper]{article}
%\usepackage[T2A]{fontenc}
\usepackage[russian]{babel}
\usepackage[utf8]{inputenc}

\usepackage{geometry}
\geometry{left=2cm}
\geometry{right=1cm}
\geometry{top=2cm}
\geometry{bottom=2cm}

%\usepackage[lutf8x]{luainputenc}
\usepackage{amsfonts}
\usepackage{amsmath}
\usepackage{epigraph}
\usepackage{appendix}
\usepackage{comment}

% misc
%\usepackage{fullpage}
\usepackage{indentfirst}
\usepackage{hyperref}

% review
\usepackage[textwidth=120]{todonotes}
\usepackage{color}

% graphics
\usepackage{graphicx}

\begin{document}    
    1. При скорости ветра $u_1 = 10$ м/с капля дождя падает под углом $30^\circ$ к вертикали. При какой скорости ветра $u 2$ капля будет падать
    под углом $45^\circ$?\\
    
    2. При взрыве покоящейся цилиндрической бомбы радиуса 50 см осколки, разлетающиеся в радиальном направлении, за время 4 с удаляются от оси цилиндра на расстояние 16 м . На какое расстояние $l$ от оси цилиндра удаляются осколки за то же время, если в момент взрыва бомба будет вращаться вокруг своей оси с угловой скоростью 6 рад/с? Влиянием силы тяжести пренебречь.\\
    
    3. В момент, когда опоздавший пассажир вбежал на платформу, мимо него прошел – за время 3 с – предпоследний вагон. Последний вагон прошел мимо пассажира за время 5 с. На сколько опоздал пассажир к отходу поезда? Поезд движется равноускоренно. Длина вагонов одинакова. \\
    
    4. Из одной точки вылетают одновременно две частицы с горизонтальными противоположно направленными
    скоростями $ v_1 = 2 $ м/с и$  v_2 = 5 $м/с. Через какой интервал времени угол между направлениями скоростей этих частиц станет равным $ 90^\circ$ ? Считать, что ускорение свободного падения $ g = 10$  м/с2 .\\
    
    5. На тело массы 1 кг, вначале покоившееся на горизонтальной плоскости, в течение времени 10 с действует горизонтальная сила 5 Н . Коэффициент трения тела о плоскость равен 0.4. Какое расстояние пройдет тело за время движения?\\
    
    6. Система из двух маятников, в которой точкой подвеса второго маятника служит массивное тело первого маятника (двойной маятник), вращается вокруг вертикальной оси так, что обе нити лежат в одной плоскости и составляют с вертикалью постоянные углы $30^\circ$ и $45^\circ$. Массы грузов маятников равны. Длины нитей одинаковы и равны 1 м. Найти угловую скорость вращения двойного маятника.\\
    
    7. Определить силу натяжения вертикального троса, медленно вытягивающего конец бревна массы m = 240 кг из воды, если бревно при вытягивании остается затопленным наполовину (т. е. происходит лишь разворот бревна вокруг его центра).\\
    
    8. Основаниями цилиндрического сосуда, из которого откачан воздух, являются две крышки площади 12 м2, каждую из которых тянут 16 лошадей. Каково ускорение лошадей в момент отрыва крышки?  Атмосферное давление равно $10^5$ Па. Масса каждой лошади равна 400 кг.\\
    
    9. Тяжелый поршень массы 10 кг вставляют в открытый сверху стоящий вертикально цилиндрический сосуд, площадь сечения которого равна 100 см2 и равна площади поршня, и отпускают. Найти давление в сосуде в момент, когда скорость поршня макси-
    мальна. Атмосферное давление равно $10^5$ Па. Трением пренебречь.\\
    
    10. В U-образной трубке с воздухом на одинаковой высоте 20 см удерживают два поршня массы 1 кг каждый. Площадь сечения левого колена трубки равна 50 см2, площадь сечения правого колена и нижней части трубки равна 25 см2. Длина нижней части равна 60 см. Давление воздуха в трубке равно атмосферному. Поршни отпускают. Найти установившиеся высоты поршней. Поршни могут перемещаться только по вертикальным участкам трубки. Температуру считать постоянной.\\
    
    11. В вертикально стоящем цилиндрическом сосуде, заполненном воздухом, находятся в равновесии два тонких одинаковых тяжелых поршня. Расстояние между поршнями и расстояние от нижнего поршня до дна сосуда одинаковы и равны l = 10 см, давление между поршнями равно удвоенному нормальному атмосферному давлению $ p_0 $ . На верхний поршень давят таким образом, что он перемещается на место нижнего. На каком расстоянии от дна будет находиться нижний поршень? Температуру воздуха считать постоянной. Трением пренебречь.\\
    
    12. В вертикальном открытом сверху цилиндрическом сосуде, имеющем площадь поперечного сечения 75 см2, на высоте 30 см от дна находится поршень массы 2,5 кг, поддерживаемый сжатым газом с молярной массой 28 гр/моль. Температура газа равна 500 К, атмосферное давление известно. Определить массу газа в сосуде. Трением пренебречь.\\
    
    
    
    
\end{document}


